% Options for packages loaded elsewhere
\PassOptionsToPackage{unicode}{hyperref}
\PassOptionsToPackage{hyphens}{url}
%
\documentclass[
]{article}
\usepackage{amsmath,amssymb}
\usepackage{iftex}
\ifPDFTeX
  \usepackage[T1]{fontenc}
  \usepackage[utf8]{inputenc}
  \usepackage{textcomp} % provide euro and other symbols
\else % if luatex or xetex
  \usepackage{unicode-math} % this also loads fontspec
  \defaultfontfeatures{Scale=MatchLowercase}
  \defaultfontfeatures[\rmfamily]{Ligatures=TeX,Scale=1}
\fi
\usepackage{lmodern}
\ifPDFTeX\else
  % xetex/luatex font selection
\fi
% Use upquote if available, for straight quotes in verbatim environments
\IfFileExists{upquote.sty}{\usepackage{upquote}}{}
\IfFileExists{microtype.sty}{% use microtype if available
  \usepackage[]{microtype}
  \UseMicrotypeSet[protrusion]{basicmath} % disable protrusion for tt fonts
}{}
\makeatletter
\@ifundefined{KOMAClassName}{% if non-KOMA class
  \IfFileExists{parskip.sty}{%
    \usepackage{parskip}
  }{% else
    \setlength{\parindent}{0pt}
    \setlength{\parskip}{6pt plus 2pt minus 1pt}}
}{% if KOMA class
  \KOMAoptions{parskip=half}}
\makeatother
\usepackage{xcolor}
\usepackage[margin=1in]{geometry}
\usepackage{graphicx}
\makeatletter
\def\maxwidth{\ifdim\Gin@nat@width>\linewidth\linewidth\else\Gin@nat@width\fi}
\def\maxheight{\ifdim\Gin@nat@height>\textheight\textheight\else\Gin@nat@height\fi}
\makeatother
% Scale images if necessary, so that they will not overflow the page
% margins by default, and it is still possible to overwrite the defaults
% using explicit options in \includegraphics[width, height, ...]{}
\setkeys{Gin}{width=\maxwidth,height=\maxheight,keepaspectratio}
% Set default figure placement to htbp
\makeatletter
\def\fps@figure{htbp}
\makeatother
\setlength{\emergencystretch}{3em} % prevent overfull lines
\providecommand{\tightlist}{%
  \setlength{\itemsep}{0pt}\setlength{\parskip}{0pt}}
\setcounter{secnumdepth}{-\maxdimen} % remove section numbering
\ifLuaTeX
  \usepackage{selnolig}  % disable illegal ligatures
\fi
\usepackage{bookmark}
\IfFileExists{xurl.sty}{\usepackage{xurl}}{} % add URL line breaks if available
\urlstyle{same}
\hypersetup{
  pdftitle={Advanced Ensemble Techniques},
  pdfauthor={Your Name},
  hidelinks,
  pdfcreator={LaTeX via pandoc}}

\title{Advanced Ensemble Techniques}
\author{Your Name}
\date{2024-07-22}

\begin{document}
\maketitle

\subsection{Introduction}\label{introduction}

In this document, we explore advanced ensemble techniques such as
XGBoost, LightGBM, and CatBoost. We will review their theoretical
foundations, implement models, and evaluate their performance on
real-world datasets.

\subsection{Theory Review}\label{theory-review}

Summarize the theoretical foundations. For instance, the boosting
process can be expressed mathematically as follows:

\[
F(x) = \sum_{m=1}^{M} \alpha_m h_m(x)
\]

where \(F(x)\) is the final model, \(\alpha_m\) is the weight of the
\(m\)-th model, and \(h_m(x)\) is the \(m\)-th weak learner.

\subsection{Implementation}\label{implementation}

\begin{itemize}
\tightlist
\item
  \textbf{XGBoost}: Implementation steps for XGBoost.
\item
  \textbf{LightGBM}: Implementation steps for LightGBM.
\item
  \textbf{CatBoost}: Implementation steps for CatBoost.
\end{itemize}

\subsection{Experimentation}\label{experimentation}

Describe the datasets and benchmarking. For instance, compare model
performance using metrics such as accuracy (\(\text{Acc}\)), precision
(\(\text{Prec}\)), and recall (\(\text{Rec}\)).

\[
\text{F1 Score} = 2 \times \frac{\text{Prec} \times \text{Rec}}{\text{Prec} + \text{Rec}}
\]

\subsection{Analysis}\label{analysis}

Evaluate efficiency and scalability. Discuss findings in terms of
training time, model accuracy, and resource utilization.

\subsection{Conclusion}\label{conclusion}

Summarize key findings, comparing the performance of XGBoost, LightGBM,
and CatBoost, and provide any conclusions.

\end{document}
